%%%%%%%%% MCI Project group21 Weekly Meeting Minutes - Week 5%%%%%%%%%%
%%%%%%%%% Author: Changchang Liu%%%%%%%%%%%%%%%%%%
%%%%%%%%% 14 March 2017 %%%%%%%%%%%%%%%%%%%

\documentclass[11pt, a4paper]{article}


\begin{document}

\title{Minutes for Third Supervisor Meeting}
\author{MCI Group21}
\date{Tuesday 14 March 2017}
\maketitle

\vspace*{15pt}

%%% title
\begin{center}
\begin{flushleft}
  \textbf{Chair}      \qquad Changchang Liu\\
  \textbf{Members}
  \begin {itemize}
  		\item a1674472 Tianxu Xi
		\item a1692335 Changchang Liu

  \end{itemize} 
  \textbf{Apologies}  \ None\\
\end{flushleft}
\end{center}



%%%% begin document

\vspace*{10pt}

%%% date, time and place
\section{Time and Place}
The \emph{third} supervisor meeting for the MCI Group Project was held in \textbf{Ingkarni Wardli level 4} at \textbf{2:30pm on Tuesday 14th March 2017}.
 
%%% quorum announcement
\section{Quorum Announcement}
The Chairperson announced that a quorum of the group was present, and that the meeting, having been duly convened, was ready to proceed with its business.



%%% minutes
\section{Pitch Presentation}

\subsection{Background}
\begin{itemize}
\item Should add the introduction about the memory trace.
\item Should add issues to be addressed.
\item Should add example for how to revise and identify errors by tracing code.
\end{itemize}

\subsection{What is the project}
\begin{itemize}
\item Delete the pictures of the already exits relative tools.
\item Add pictures about how the interface of out tool will be liked.

\end{itemize}

\subsection{Current tools}
\begin{itemize}
\item Delete the slide of current tools.

\end{itemize}

\subsection{Advantage}
\begin{itemize}
\item  Add the common errors need revision.
\end{itemize}

\subsection{Future Improvement}
\begin{itemize}
\item Delete the item of provide teaching plan for teachers.
\item Add provide hints.
\end{itemize}


\section{Project develop issues}
\subsection{Trace table design}
\begin{itemize}
\item Need to try resetting the structure of the button.
\item Need to learn the relative techniques.
\end{itemize}

\subsection{Shell script for gdb}
\begin{itemize}
\item Try "grep".
\item Using shell script and learn from the relative techniques.
\end{itemize}

\subsection{Interface design}
\begin{itemize}
\item Allowed to use the interface sample.
\end{itemize}

\subsection{Database design}
\begin{itemize}
\item Need one database to store login information.
\item Need one database to store C code and its solution.
\item Need one database to store students' marks.
\end{itemize}




%%% action to take
\section{Action Plan}
We will more focus on how to generate the correct answers and how to compare with student's answers at this stage.

\vspace{10pt}
%%% action plan
\begin{tabular}{|c|c|c|c|}
\hline
No. & Action item & Owner & Deadline \\
\hline
\hline
1 & Change the context of the pitch presentation slides & all & 16 March 2017 \\
\hline
2 & Prepare for the pitch presentation & all & 16 March 2017 \\
\hline
3 & Further generate the correct answer and can run by shell script & Changchang & 21 March 2017 \\
\hline
4 & Store the correct answer into database & all & 23 March 2017\\
\hline
\end{tabular}

%%% adjournment
\section{Adjournment}
The next meeting is a \emph{Supervisor} meeting with all members present, and will be held in \emph{Ingkarni Wardli Level 4} at \emph{2:30pm} on \emph{21 March, 2017}.



\end{document}

