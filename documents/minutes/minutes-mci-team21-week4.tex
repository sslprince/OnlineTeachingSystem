%%%%%%%%% MCI Project group21 Weekly Meeting Minutes - Week 5%%%%%%%%%%
%%%%%%%%% Author: Changchang Liu%%%%%%%%%%%%%%%%%%
%%%%%%%%% 14 March 2017 %%%%%%%%%%%%%%%%%%%

\documentclass[11pt, a4paper]{article}


\begin{document}

\title{Minutes of 4th Meeting with Supervisor}
\author{MCI Team21}
\date{Tuesday 21 March 2017}
\maketitle

\vspace*{15pt}

%%% title
\begin{center}
\begin{flushleft}
  \textbf{Chair}      \qquad Tianxu Xi\\
  \textbf{Secretary}   \qquad Tianxu Xi\\
  \textbf{Group Members}
  \begin {itemize}
  		\item a1674472 Tianxu Xi
		\item a1692335 Changchang Liu

  \end{itemize} 
  \textbf{Apologies}  \ None\\
\end{flushleft}
\end{center}



%%%% begin document

\vspace*{10pt}

%%% date, time and place
\section{Time and Place}
The \emph{4th} group meeting with supervisor for the MCI Group Project was held in \textbf{Ingkarni Wardli level 4} at \textbf{2:30pm on Tuesday 21th March 2017}.
 
%%% quorum announcement
\section{Quorum Announcement}
The Chairperson announced that a quorum of the group was present, and that the meeting, having been duly convened, was ready to proceed with its business.

\section{Summary of previous meeting}
We showed our pitch power-point to supervisor. Cruz pointed some significant parts that we need to change and also told us the incorrect structure for ppt. In addition, we discussed the design of trace table, web interface for clients, how to work out for the gdb and how to use database for our tool.

\section{Pitch Presentation}
%%% minutes
\subsection{Background}
\begin{itemize}
\item we need to introduce what the trace table is clearly.
\item point the function of trace table: show the value of variables step by step. 
\item pay more attention on the advantages of it. Tell clients it will decrease the paperwork. From paper to program, it will become more easier to practice and manage.
\item For the example after this slide, we just need to mention one statements to show how the trace table works.

\end{itemize}

\subsection{Introduction of this project}
\begin{itemize}
\item Follow the structure that we have.
\item Maybe put the feedback parts into the extension.

\end{itemize}

\subsection{Current tools}
\begin{itemize}
\item For current paperwork, students need to use pen to write it down and teacher also need to correct it manually. It is not a effective way.
\item Need to be changed. Show clients the current tool of ours, not for others. Point out that it is just a simple one, we need to improve it, not only expression and loops. Tell clients the progress for the current one.

\end{itemize}

\subsection{Advantage}
\begin{itemize}
\item Students can only type some numbers for filling in the trace table and they will receive the feedbacks and hints automatically.
\item Teacher can collect the marks of students by clicking one or two buttons. In addition, this tools will give teacher the information about students' overall programming level.
\end{itemize}

\subsection{Future Improvement}
\begin{itemize}
\item Delete the repeated sentences about different programming languages.
\item This tool should show students some more effective hints and give them some more extension practice.
\item This tool should show teacher the trend of programming level in one class.
\item We need to beautify our interface for clients.
\item This tool should show teacher the teaching plan according to the marks of students.
\end{itemize}


\section{Project develop issues}
\subsection{Trace table design}
\begin{itemize}
\item Need to try resetting the structure of the button.
\item Need to learn the relative techniques.
\end{itemize}

\subsection{Shell script for gdb}
\begin{itemize}
\item Already have some achievements for this parts and maybe need to change the format of output file.
\end{itemize}

\subsection{Interface design}
\begin{itemize}
\item Need to work out the interface of teacher mode. Teachers can choose the class firstly, after that, the tool will show them the students in this class.
\item For teacher mode, we need to change page structure, do not put everything into one page. We need to create some pop-up pages to show clients detail information.
\item still need to think about how to design student mode interface.
\end{itemize}

\subsection{Database design}
\begin{itemize}
\item Still need to work out how to storage data and compare the results.
\end{itemize}




%%% action to take
\section{Action Plan}
We will more focus on how to generate the correct answers and how to compare with student's answers at this stage.

\vspace{10pt}
%%% action plan
\begin{tabular}{|c|c|c|c|}
\hline
No. & Action item & Owner & Deadline \\
\hline
\hline
1 & Change the context of the pitch presentation slides & all & 21 March 2017 \\
\hline
2 & Finish login part & all & 23 March 2017 \\
\hline
3 & Finish design about student interface & all & 24 March 2017 \\
\hline
4 & Store the correct answer into database and compare them & all & 26 March 2017\\
\hline
\end{tabular}

%%% adjournment
\section{Adjournment}
The next meeting is a \emph{Supervisor} meeting with all members present, and will be held in \emph{Ingkarni Wardli Level 4} at \emph{2:30pm} on \emph{28 March, 2017}.



\end{document}
