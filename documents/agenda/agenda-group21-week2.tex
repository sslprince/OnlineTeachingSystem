%%%%%%%%% MCI team21 Group Meeting - Week 1 %%%%%%%%
%%%%%%%%% Author: Changchang Liu %%%%%%%%%%%%%%%%%%%
%%%%%%%%% 28 February 2017 %%%%%%%%%%%%%%%%%%%


\documentclass[11pt, a4paper]{article}


\begin{document}


\title{Minutes of MCI team21 Group Meeting}
\date{Week 2, Tuesday 7 March 2017}
\maketitle

\vspace*{15pt}

%%% title
\begin{center}
\begin{flushleft}
  \textbf{Chair}      \qquad Tianxu Xi\\
  \textbf{Secretary}  \qquad Tianxu Xi\\
  \textbf{Members}    \qquad Cruz Izu, Changchang Liu, Tianxu Xi\\
  \textbf{Apologies}  \qquad None \\
\end{flushleft}
\end{center}



%%%% begin document

\vspace*{10pt}

%%% time and place
\section{Time and Place}
The second group meeting in Week 2 for the MCI Project was held on \textbf{Level 4 of Ingkarni Wardli} at \textbf{2:30pm} on \textbf{Tuesday 7 March 2017}. \\
 
%%% meeting agenda
\section{Quorum Announcement}

The Chairman announced that a quorum of the group was present, and that the
meeting, having been duly convened, was ready to proceed with its business.

\section{ Summary of previous meeting}

We did brief introduction of ourselves, had general understanding about this tool and asked some questions about how we can achieve this tools and some details.  

%%% minutes
\section{Minutes}

\subsection{Group Github Repository}

The team still need to update our github repository and upload our work frequently. \\

\subsection{Responsiabilities and workload}

Tianxu will still work on php and the display part, Changchang should find how we can use debugger to extract all variables and their values. Besides, we also need to check and help others work and try to connect our work together.\\


\subsection{Requirements Discussion}

We need to find a good tool like debugger to help us get the variables in a program, because our program is complex and hard to optimize.\\ 

XGDB is a good one that we should check, because we can create breakpoint to see every step. Besides, it is better for us to search if there is a greater tool that can make our work easier.\\

It is also a good way to put all variables of every line into a excel file. It could be kind of a form or table, so it is easier for us to compare them with the values that students fill in. We need to think about this method.\\

For php, we need to think about how to use "echo" to show the comments into the input text field. Also we should figure out how to add a new line in the form by clicking "add" button.\\

It is a good news that we can use "diff" to compare every line in two files and see which lines are different. This is a significant point to check which parts that student fill in are not correct .\\

We should create a login interface for students and teacher. Also, the database is necessary for this tool, such as MySQL, because this tool should help teacher collect information of all students. However, we will design this tool for one student at the beginning.






%%% next meeting
\section{Next Meeting}
The next meeting is a \emph{client} meeting with all members present, and will be held in \emph{Ingkarni Wardli } at \emph{2:30pm} on \emph{Tuesday 14th March, 2017}.







\end{document}